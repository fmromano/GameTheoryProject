
%% bare_conf.tex
%% V1.3
%% 2007/01/11
%% by Michael Shell
%% See:
%% http://www.michaelshell.org/
%% for current contact information.
%%
%% This is a skeleton file demonstrating the use of IEEEtran.cls
%% (requires IEEEtran.cls version 1.7 or later) with an IEEE conference paper.
%%
%% Support sites:
%% http://www.michaelshell.org/tex/ieeetran/
%% http://www.ctan.org/tex-archive/macros/latex/contrib/IEEEtran/
%% and
%% http://www.ieee.org/

%%*************************************************************************
%% Legal Notice:
%% This code is offered as-is without any warranty either expressed or
%% implied; without even the implied warranty of MERCHANTABILITY or
%% FITNESS FOR A PARTICULAR PURPOSE! 
%% User assumes all risk.
%% In no event shall IEEE or any contributor to this code be liable for
%% any damages or losses, including, but not limited to, incidental,
%% consequential, or any other damages, resulting from the use or misuse
%% of any information contained here.
%%
%% All comments are the opinions of their respective authors and are not
%% necessarily endorsed by the IEEE.
%%
%% This work is distributed under the LaTeX Project Public License (LPPL)
%% ( http://www.latex-project.org/ ) version 1.3, and may be freely used,
%% distributed and modified. A copy of the LPPL, version 1.3, is included
%% in the base LaTeX documentation of all distributions of LaTeX released
%% 2003/12/01 or later.
%% Retain all contribution notices and credits.
%% ** Modified files should be clearly indicated as such, including  **
%% ** renaming them and changing author support contact information. **
%%
%% File list of work: IEEEtran.cls, IEEEtran_HOWTO.pdf, bare_adv.tex,
%%                    bare_conf.tex, bare_jrnl.tex, bare_jrnl_compsoc.tex
%%*************************************************************************

% *** Authors should verify (and, if needed, correct) their LaTeX system  ***
% *** with the testflow diagnostic prior to trusting their LaTeX platform ***
% *** with production work. IEEE's font choices can trigger bugs that do  ***
% *** not appear when using other class files.                            ***
% The testflow support page is at:
% http://www.michaelshell.org/tex/testflow/



% Note that the a4paper option is mainly intended so that authors in
% countries using A4 can easily print to A4 and see how their papers will
% look in print - the typesetting of the document will not typically be
% affected with changes in paper size (but the bottom and side margins will).
% Use the testflow package mentioned above to verify correct handling of
% both paper sizes by the user's LaTeX system.
%
% Also note that the "draftcls" or "draftclsnofoot", not "draft", option
% should be used if it is desired that the figures are to be displayed in
% draft mode.
%
\documentclass[conference]{IEEEtran}
% Add the compsoc option for Computer Society conferences.
%
% If IEEEtran.cls has not been installed into the LaTeX system files,
% manually specify the path to it like:
% \documentclass[conference]{../sty/IEEEtran}





% Some very useful LaTeX packages include:
% (uncomment the ones you want to load)


% *** MISC UTILITY PACKAGES ***
%
%\usepackage{ifpdf}
% Heiko Oberdiek's ifpdf.sty is very useful if you need conditional
% compilation based on whether the output is pdf or dvi.
% usage:
% \ifpdf
%   % pdf code
% \else
%   % dvi code
% \fi
% The latest version of ifpdf.sty can be obtained from:
% http://www.ctan.org/tex-archive/macros/latex/contrib/oberdiek/
% Also, note that IEEEtran.cls V1.7 and later provides a builtin
% \ifCLASSINFOpdf conditional that works the same way.
% When switching from latex to pdflatex and vice-versa, the compiler may
% have to be run twice to clear warning/error messages.






% *** CITATION PACKAGES ***
%
%\usepackage{cite}
% cite.sty was written by Donald Arseneau
% V1.6 and later of IEEEtran pre-defines the format of the cite.sty package
% \cite{} output to follow that of IEEE. Loading the cite package will
% result in citation numbers being automatically sorted and properly
% "compressed/ranged". e.g., [1], [9], [2], [7], [5], [6] without using
% cite.sty will become [1], [2], [5]--[7], [9] using cite.sty. cite.sty's
% \cite will automatically add leading space, if needed. Use cite.sty's
% noadjust option (cite.sty V3.8 and later) if you want to turn this off.
% cite.sty is already installed on most LaTeX systems. Be sure and use
% version 4.0 (2003-05-27) and later if using hyperref.sty. cite.sty does
% not currently provide for hyperlinked citations.
% The latest version can be obtained at:
% http://www.ctan.org/tex-archive/macros/latex/contrib/cite/
% The documentation is contained in the cite.sty file itself.






% *** GRAPHICS RELATED PACKAGES ***
%
\ifCLASSINFOpdf
 \usepackage[pdftex]{graphicx}
  % declare the path(s) where your graphic files are
  % \graphicspath{{../pdf/}{../jpeg/}}
  % and their extensions so you won't have to specify these with
  % every instance of \includegraphics
  % \DeclareGraphicsExtensions{.pdf,.jpeg,.png}
\else
  % or other class option (dvipsone, dvipdf, if not using dvips). graphicx
  % will default to the driver specified in the system graphics.cfg if no
  % driver is specified.
  % \usepackage[dvips]{graphicx}
  % declare the path(s) where your graphic files are
  % \graphicspath{{../eps/}}
  % and their extensions so you won't have to specify these with
  % every instance of \includegraphics
  % \DeclareGraphicsExtensions{.eps}
\fi
% graphicx was written by David Carlisle and Sebastian Rahtz. It is
% required if you want graphics, photos, etc. graphicx.sty is already
% installed on most LaTeX systems. The latest version and documentation can
% be obtained at: 
% http://www.ctan.org/tex-archive/macros/latex/required/graphics/
% Another good source of documentation is "Using Imported Graphics in
% LaTeX2e" by Keith Reckdahl which can be found as epslatex.ps or
% epslatex.pdf at: http://www.ctan.org/tex-archive/info/
%
% latex, and pdflatex in dvi mode, support graphics in encapsulated
% postscript (.eps) format. pdflatex in pdf mode supports graphics
% in .pdf, .jpeg, .png and .mps (metapost) formats. Users should ensure
% that all non-photo figures use a vector format (.eps, .pdf, .mps) and
% not a bitmapped formats (.jpeg, .png). IEEE frowns on bitmapped formats
% which can result in "jaggedy"/blurry rendering of lines and letters as
% well as large increases in file sizes.
%
% You can find documentation about the pdfTeX application at:
% http://www.tug.org/applications/pdftex





% *** MATH PACKAGES ***
%
\usepackage[cmex10]{amsmath}
% A popular package from the American Mathematical Society that provides
% many useful and powerful commands for dealing with mathematics. If using
% it, be sure to load this package with the cmex10 option to ensure that
% only type 1 fonts will utilized at all point sizes. Without this option,
% it is possible that some math symbols, particularly those within
% footnotes, will be rendered in bitmap form which will result in a
% document that can not be IEEE Xplore compliant!
%
% Also, note that the amsmath package sets \interdisplaylinepenalty to 10000
% thus preventing page breaks from occurring within multiline equations. Use:
%\interdisplaylinepenalty=2500
% after loading amsmath to restore such page breaks as IEEEtran.cls normally
% does. amsmath.sty is already installed on most LaTeX systems. The latest
% version and documentation can be obtained at:
% http://www.ctan.org/tex-archive/macros/latex/required/amslatex/math/





% *** SPECIALIZED LIST PACKAGES ***
%
%\usepackage{algorithmic}
\usepackage{algpseudocode}
\usepackage[]{algorithm2e}
% algorithmic.sty was written by Peter Williams and Rogerio Brito.
% This package provides an algorithmic environment fo describing algorithms.
% You can use the algorithmic environment in-text or within a figure
% environment to provide for a floating algorithm. Do NOT use the algorithm
% floating environment provided by algorithm.sty (by the same authors) or
% algorithm2e.sty (by Christophe Fiorio) as IEEE does not use dedicated
% algorithm float types and packages that provide these will not provide
% correct IEEE style captions. The latest version and documentation of
% algorithmic.sty can be obtained at:
% http://www.ctan.org/tex-archive/macros/latex/contrib/algorithms/
% There is also a support site at:
% http://algorithms.berlios.de/index.html
% Also of interest may be the (relatively newer and more customizable)
% algorithmicx.sty package by Szasz Janos:
% http://www.ctan.org/tex-archive/macros/latex/contrib/algorithmicx/




% *** ALIGNMENT PACKAGES ***
%
\usepackage{array}
% Frank Mittelbach's and David Carlisle's array.sty patches and improves
% the standard LaTeX2e array and tabular environments to provide better
% appearance and additional user controls. As the default LaTeX2e table
% generation code is lacking to the point of almost being broken with
% respect to the quality of the end results, all users are strongly
% advised to use an enhanced (at the very least that provided by array.sty)
% set of table tools. array.sty is already installed on most systems. The
% latest version and documentation can be obtained at:
% http://www.ctan.org/tex-archive/macros/latex/required/tools/
%\usepackage{booktabs}


\usepackage{mdwmath}
\usepackage{mdwtab}
% Also highly recommended is Mark Wooding's extremely powerful MDW tools,
% especially mdwmath.sty and mdwtab.sty which are used to format equations
% and tables, respectively. The MDWtools set is already installed on most
% LaTeX systems. The lastest version and documentation is available at:
% http://www.ctan.org/tex-archive/macros/latex/contrib/mdwtools/


% IEEEtran contains the IEEEeqnarray family of commands that can be used to
% generate multiline equations as well as matrices, tables, etc., of high
% quality.


%\usepackage{eqparbox}
% Also of notable interest is Scott Pakin's eqparbox package for creating
% (automatically sized) equal width boxes - aka "natural width parboxes".
% Available at:
% http://www.ctan.org/tex-archive/macros/latex/contrib/eqparbox/





% *** SUBFIGURE PACKAGES ***
%\usepackage[tight,footnotesize]{subfigure}
% subfigure.sty was written by Steven Douglas Cochran. This package makes it
% easy to put subfigures in your figures. e.g., "Figure 1a and 1b". For IEEE
% work, it is a good idea to load it with the tight package option to reduce
% the amount of white space around the subfigures. subfigure.sty is already
% installed on most LaTeX systems. The latest version and documentation can
% be obtained at:
% http://www.ctan.org/tex-archive/obsolete/macros/latex/contrib/subfigure/
% subfigure.sty has been superceeded by subfig.sty.



%\usepackage[caption=false]{caption}
%\usepackage[font=footnotesize]{subfig}
% subfig.sty, also written by Steven Douglas Cochran, is the modern
% replacement for subfigure.sty. However, subfig.sty requires and
% automatically loads Axel Sommerfeldt's caption.sty which will override
% IEEEtran.cls handling of captions and this will result in nonIEEE style
% figure/table captions. To prevent this problem, be sure and preload
% caption.sty with its "caption=false" package option. This is will preserve
% IEEEtran.cls handing of captions. Version 1.3 (2005/06/28) and later 
% (recommended due to many improvements over 1.2) of subfig.sty supports
% the caption=false option directly:
%\usepackage[caption=false,font=footnotesize]{subfig}
%
% The latest version and documentation can be obtained at:
% http://www.ctan.org/tex-archive/macros/latex/contrib/subfig/
% The latest version and documentation of caption.sty can be obtained at:
% http://www.ctan.org/tex-archive/macros/latex/contrib/caption/




% *** FLOAT PACKAGES ***
%
%\usepackage{fixltx2e}
% fixltx2e, the successor to the earlier fix2col.sty, was written by
% Frank Mittelbach and David Carlisle. This package corrects a few problems
% in the LaTeX2e kernel, the most notable of which is that in current
% LaTeX2e releases, the ordering of single and double column floats is not
% guaranteed to be preserved. Thus, an unpatched LaTeX2e can allow a
% single column figure to be placed prior to an earlier double column
% figure. The latest version and documentation can be found at:
% http://www.ctan.org/tex-archive/macros/latex/base/



%\usepackage{stfloats}
% stfloats.sty was written by Sigitas Tolusis. This package gives LaTeX2e
% the ability to do double column floats at the bottom of the page as well
% as the top. (e.g., "\begin{figure*}[!b]" is not normally possible in
% LaTeX2e). It also provides a command:
%\fnbelowfloat
% to enable the placement of footnotes below bottom floats (the standard
% LaTeX2e kernel puts them above bottom floats). This is an invasive package
% which rewrites many portions of the LaTeX2e float routines. It may not work
% with other packages that modify the LaTeX2e float routines. The latest
% version and documentation can be obtained at:
% http://www.ctan.org/tex-archive/macros/latex/contrib/sttools/
% Documentation is contained in the stfloats.sty comments as well as in the
% presfull.pdf file. Do not use the stfloats baselinefloat ability as IEEE
% does not allow \baselineskip to stretch. Authors submitting work to the
% IEEE should note that IEEE rarely uses double column equations and
% that authors should try to avoid such use. Do not be tempted to use the
% cuted.sty or midfloat.sty packages (also by Sigitas Tolusis) as IEEE does
% not format its papers in such ways.





% *** PDF, URL AND HYPERLINK PACKAGES ***
%
%\usepackage{url}
% url.sty was written by Donald Arseneau. It provides better support for
% handling and breaking URLs. url.sty is already installed on most LaTeX
% systems. The latest version can be obtained at:
% http://www.ctan.org/tex-archive/macros/latex/contrib/misc/
% Read the url.sty source comments for usage information. Basically,
% \url{my_url_here}.





% *** Do not adjust lengths that control margins, column widths, etc. ***
% *** Do not use packages that alter fonts (such as pslatex).         ***
% There should be no need to do such things with IEEEtran.cls V1.6 and later.
% (Unless specifically asked to do so by the journal or conference you plan
% to submit to, of course. )

%Examples of how to insert figures and subsections

% An example of a floating figure using the graphicx package.
% Note that \label must occur AFTER (or within) \caption.
% For figures, \caption should occur after the \includegraphics.
% Note that IEEEtran v1.7 and later has special internal code that
% is designed to preserve the operation of \label within \caption
% even when the captionsoff option is in effect. However, because
% of issues like this, it may be the safest practice to put all your
% \label just after \caption rather than within \caption{}.
%
% Reminder: the "draftcls" or "draftclsnofoot", not "draft", class
% option should be used if it is desired that the figures are to be
% displayed while in draft mode.
%
%\begin{figure}[!t]
%\centering
%\includegraphics[width=2.5in]{myfigure}
% where an .eps filename suffix will be assumed under latex, 
% and a .pdf suffix will be assumed for pdflatex; or what has been declared
% via \DeclareGraphicsExtensions.
%\caption{Simulation Results}
%\label{fig_sim}
%\end{figure}

% Note that IEEE typically puts floats only at the top, even when this
% results in a large percentage of a column being occupied by floats.


% An example of a double column floating figure using two subfigures.
% (The subfig.sty package must be loaded for this to work.)
% The subfigure \label commands are set within each subfloat command, the
% \label for the overall figure must come after \caption.
% \hfil must be used as a separator to get equal spacing.
% The subfigure.sty package works much the same way, except \subfigure is
% used instead of \subfloat.
%
%\begin{figure*}[!t]
%\centerline{\subfloat[Case I]\includegraphics[width=2.5in]{subfigcase1}%
%\label{fig_first_case}}
%\hfil
%\subfloat[Case II]{\includegraphics[width=2.5in]{subfigcase2}%
%\label{fig_second_case}}}
%\caption{Simulation results}
%\label{fig_sim}
%\end{figure*}
%
% Note that often IEEE papers with subfigures do not employ subfigure
% captions (using the optional argument to \subfloat), but instead will
% reference/describe all of them (a), (b), etc., within the main caption.


% An example of a floating table. Note that, for IEEE style tables, the 
% \caption command should come BEFORE the table. Table text will default to
% \footnotesize as IEEE normally uses this smaller font for tables.
% The \label must come after \caption as always.
%
%\begin{table}[!t]
%% increase table row spacing, adjust to taste
%\renewcommand{\arraystretch}{1.3}
% if using array.sty, it might be a good idea to tweak the value of
% \extrarowheight as needed to properly center the text within the cells
%\caption{An Example of a Table}
%\label{table_example}
%\centering
%% Some packages, such as MDW tools, offer better commands for making tables
%% than the plain LaTeX2e tabular which is used here.
%\begin{tabular}{|c||c|}
%\hline
%One & Two\\
%\hline
%Three & Four\\
%\hline
%\end{tabular}
%\end{table}

%Example formating
%\hfill mds
%
%\hfill January 11, 2007
%
%\subsection{Subsection Heading Here}
%Subsection text here.
%
%
%\subsubsection{Subsubsection Heading Here}
%Subsubsection text here.

%End of the examples (I wanted them out of the main paper)


% Note that IEEE does not put floats in the very first column - or typically
% anywhere on the first page for that matter. Also, in-text middle ("here")
% positioning is not used. Most IEEE journals/conferences use top floats
% exclusively. Note that, LaTeX2e, unlike IEEE journals/conferences, places
% footnotes above bottom floats. This can be corrected via the \fnbelowfloat
% command of the stfloats package.


% correct bad hyphenation here
\hyphenation{op-tical net-works semi-conduc-tor}


\begin{document}
%
% paper title
% can use linebreaks \\ within to get better formatting as desired
\title{Matching Theory and Virtual Machines}


% author names and affiliations
% use a multiple column layout for up to three different
% affiliations
\author{\IEEEauthorblockN{Kristen Hines}
\IEEEauthorblockA{School of Electrical and\\Computer Engineering\\
Virginia Tech\\
Email: kphines@vt.edu}
\and
\IEEEauthorblockN{Ferdinando Romano}
\IEEEauthorblockA{School of Electrical and\\Computer Engineering\\
Virginia Tech\\
Email: fmromano@vt.edu}}


% Make the title area
\maketitle


%Abstract of the paper.
\begin{abstract}
%\boldmath
The assignment of jobs to separate compute clusters 
can be approached using matching theory. 
The problem is modeled as a modified college admissions
game where each institution has multiple quotas,
each of which is of a specified type.
An algorithm based on multiple iterations of the 
colleged admissions deferred acceptance algorithm is 
proposed.
The algorithm is shown to terminate, result in 
a stable mathcing, and, under certain common conditions,
approximate an optimal solution.
The proposed algorithm is simulated 
and shown to result in either significant improvement 
or only minor regression compared to other approaches.
\end{abstract}


\section{Introduction}
Cloud computing is a system that allows ubiquitous IT services,
ranging from online social networking services to infrastructure
outsources.  This is because cloud computing is expected to
be a cost-effective and flexible way to handle data and programs.
These cloud computing services are packaged in the form of
virtual machines through the use of virtualization technology.
Virtualization technology allows for computing technologies
to be virtualized by emulating processors, main memory, storage,
and networking devices [resource needed].  The final product after this virtualization
is the virtual machine.

One of the benefits of virtual machines, especially ubiquitous
ones, is they can be configured to suit a specific application's
needs, such as application isolation, security requirements,
service level-agreements, and computational performance.  [resource needed]
These virtual machines and cloud computing servers
are still housed on powerful physical machines at this point.
[need to figure out the logic between here and our problem...]

This project is focused on turning a simple virtual machine job assignment
problem into a college admissions game.  The jobs will be the applicants,
and the virtual machines will be institutions.  The jobs will apply for spots
on the virtual machines until either there are no more jobs in the 
queue, or there are no more virtual machines created and availabe for the
users at this time.  It is assumed that the virtual machines take a long
enough time to create that a user will not want to just make one
once all of the already made virtual machines are taken.  In this paper,
a computer-optimal algorithm will be proposed to provide a solution
that is better for both the virtual machines and the jobs.  More will be stated
in section three, Method Description.

The paper will proceed as follows.  The second section will cover the
background knowledge for matching games and work that is 
related to virtual machines and cloud computing.  The third section
will cover the problem formulation and proposed algorithm.  The 
fourth section will cover the results.  The fifth section will
be the conclusion of the work here.  The sixth section will
give possible future routes this work can take.  

\section{Background}
A review about the stable marriage problem is needed
to help the reader understand the proposed solution.
The stable marriage problem is a one-to-one matching
model used to effectively match two groups of agents
 together, such as men and women for a marriage.  
These agents are two disjoint sets.  Each agent
has a complete, strict, and  transitive preference over other
individuals, which means the agent is not indifferent for any choices.  In 
addition to this quality, each agent has a chance
of being unmatched. For demonstration purposes,
the two agent sets are men and women, whose sets are 
\( \emph{W} = \{ {w_1}, {w_2}, \ldots {w_p} \} \) and
\( \emph{M} =\{ {m_1}, {m_2}, \ldots {m_n} \} \), 
where \emph{p} does not have to equal \emph{n}.

Their preferences are arranged and represented as ranked 
ordered lists.  An example of such a list is
\( {p_m}_n ={w_2}, {w_4}, \ldots, \emptyset \),
where \( {w_2} \) is the man's first choice for a partner, 
\( {w_4} \) is the second, and so on.  In this case,the final 
choice represents when the man prefers to be single over 
his possible choices.

\emph{Definition 1}:  An outcome is a matching
 \( \mu \) : \emph{M} \( \times \)  \emph{W} \( \rightarrow \)  \emph{M} \( \times \)  \emph{W} such that
\emph{w} = \( \mu(m) \) if and only if 
\emph{m} = \( \mu(w) \) \( \in W \cup \emptyset \), \( \mu(m) \in M \cup \emptyset \)
for all \emph{m, w}.

This implies that agents from one set are matched to either the agents
of the other set or to the null set.  Agents' preferences over outcomes are
determined only by their own preferences for certain partners.

\emph{Definition 2}:  A matching \( \mu \) is stable if and only if it is individual rational and
not black by any pair of agents.

Individual rational and blocking pair are defined as followed.

\emph{Definition 3}:  A matiching set is individual rational to all agents if and only if
there does not exist an agent \emph{i} who prefers being unmatched to being
matched.

\emph{Definition 4}:  A matching \( \mu \) is blocked by a pair of agents \emph{(m,w)} if they
prefer each other to the partner they receive at u.  That is, \emph{w} \(\succ_m \) \( \mu (m) \)
and \emph{m} \(\succ_w \) \( \mu (w) \).  Such a pair is called a blocking set,
where \(\succ \) represents an agent's preference of one individual over another.

This means that as long as a matching is not blocked and a matching set is individually
rational, a matched set will be stable.  Therefore:

\emph{Theorem 1}:  A stable matching pair exists for every marriage market.

This theorem was proposed and proved by Gale and Shapley
by using their deferred acceptance algorithm, which was proven in [1].

The college applications game extends the concepts behind stable
marriage.  For a college applications game, the two sets of players are the
institutions and the applicants.  Institutions can be matched to multiple
applicants at one time.

\emph{Theorem 2}:  Every applicant is at least as well off under the assignment
given by the deferred acceptance procedure as he would be under any other
stable arrangement.

The end result will be a stable and optimal pairing between
 the institutions and the applicants.  This is also proven in [1].  One 
of the issues with the stable marriage problem and the college admissions problem
is that they are applicant optimal.  In other works, the set who is being
applied to, such as the women in the earlier example, are not guaranteed
their optimal choices.  If the men and women switched, and the
women were the ones proposing to the men, the men would
not be guaranteed their optimal choices, either.

\section{Method Description}
The problem formulation and proposed algorithm will be
described in this section.

\subsection{Problem Formulation}
The problem this paper is exploring is how to optimize job 
assigment to separate computer clusters.  Each of these
jobs perform differently on different cores types.  These
core types can be graphical processors, computational
processors, or something else.  Each of these clusters
have different core types.  Each job can only be divided into
a finite number of threads, and each job is assigned to 
one computer at a time.

This problem can be formed as a college admissions
game.  The differences are that the institutions
have multiple quotas, each applicant can fill multiple
slots of different types, applicants prefer different
slot types over others, and an applicant cannot be
divided among multiple institutions.

Key assumptions for this problem are:  the virtual machines
will be treated like flexible computers, jobs are submitted at
the same time, chosen jobs are completed simulatenously,
unchosen jobs will be submitted with the next round, no
indifference, and no externalities.

\subsection{Proposed Algorithm}
Things to do:  Write the math and assumptions, state why
we are doing this, and show that this will be stable as well.

The proposed algorithm is based on a deferred acceptance
college admissions algorithm with special modifications to optimize it for 
the situation where the institutions have multiple quotas of different
types and the applicants can fill multiple slots.

\subsubsection{Algorithm}
\begin{algorithm}
\caption{Proposed Algorithm}\label{alg:PA}
\KwData{Number of cores per VM available, Speed ratio matrix for jobs, Max threads used per job}
\KwResult{Matrix with jobs matched to VMs}

Initialize preference matrices and quotas\;

\While{Either jobs or VMs are left}{
	1. Set quota to 1 for each non-full VM\;
	2. Run Deferred Acceptance Algorithm\;
	3. Choose job who used the most of the computer resources and keep it\;
	4. Save this job to the results matrix\;
	5. Update preference matrices and core availability matrix\;
}

Compile results and send to user\
\end{algorithm}


\underline{Step 1}: Calculate the preferences of the jobs (applicants) and the computers
(institutions).
A job's preference for a particular computer is determined simply:
given the processors available on each computer, if a job would peform 
faster on one computer than another, then the first computer is preferred over the second. 
Given the relative speeds of each processor at performing a particular job,
the speed of that job on a particular computer is calculated by first choosing the fastest
available processors until the job's processor limit is reached or there are no more
processors available on the computer.
Then, the speeds of the chosen processors are summed together to give 
the computer's total speed at the job.
A job prefers one computer over another if its total speed
is higher than the other's. 

A computer's preference for a particular job is based on the assumption
that a computer wants to maximize utilization of its resources. 
In this case, a job that can utilize more processors than another
is preferrable. It is assumed that the number of processors that a job can use 
does not depend on the computer or the processor types and so each computer
has the same preference ranking of jobs.


\underline{Step 2}: Perform a 1-1 Matching.
The jobs are matched to the computers according the calculated preferences
using a college admissions algorithm where the quota of a computer is 
set to 1 if it has at least 1 processor still available. 
Otherwise, its quota is set to 0.

\underline{Step 3}: Determine the most important matching.
A job that can use the greatest number of processors is the
most highly preferred and so it is matched to its first choice
of computer. Thus this pair is stable and can be assigned to 
the finalized matching of the algorithm. 
This job and the processors it uses are no longer available
so they are removed from future iterations of the algorithm.

\underline{Step 4}: Return to Step 1.
The algorithm is repeated either until all processors are 
assigned a job or until all jobs are matched 
to a computer.

\subsubsection{Guaranteed of Termination}
The algorithm is guaranteed to terminate because there is a 
finite number of jobs and each iteration of the algorithm
matches one job to a computer.

\subsubsection{Stability of Algorithm}
The proposed algorithm produces a stable matching because in
each iteration, the college admissions game is used to find
a set of stable matchings.
Out of the jobs listed in the resultant set of stable matchings,
the job that can use the most processors is preferred most
by every computer. Thus, that job will be matched with its first 
choice and its matching to a computer is a stable matching.
Thus, each pair produced by an iteration of the proposed
algorithm is stable and therefore the final matching is stable.

\subsubsection{Optimality of Algorithm}
Whether the matching is optimal can be understood in multiple senses.
In this section, three different approaches to optimality are 
discussed as they apply to the proposed algorithm.

\underline{Resource Utilization}:
A simple goal of the proposed algorithm would be to 
maximize processor utiliztion so that no computing 
resources go unused/wasted.

The proposed algorithm does not alway maximize processor 
utilization. However, it does in every iteration 
where the preferred computer of the job with
the greatest possible processor utilization
has at least as many processors 
available as either
    i) that job can use or
    ii) any other computer has.
This situation is common because, often, a computer with more 
available processors will outperform
one with fewer. The exceptions occurs where there is 
a computer that has special purpose procesors
that significantly outperform those available at other 
computers and this computer does not meet either 
of conditions i) or ii) listed above.

\underline{Total Job Completion Time}:
The total computation time, i.e., the 
sum of total computation times for each job,
is another good measure of the optimality 
of the proposed algorithm.

Assuming individual jobs cannot take advantage
of processors previously used by other jobs
that have completed, 
the proposed algorithm minimizes total 
computation time whenever 
jobs that use more processors are jobs that would take 
longer to complete than any other job. 
By 'take longer to complete', we mean take longer than
other jobs if the other jobs were to use a subset or 
superset of the processors used by the first job.
When this condition is met, the job that takes the
longest is given the greatest speed possible, the job that 
takes the 2nd longest is given the next greatest speed 
possible for it, and so on. 
Thus, total computation time is minimized.

In the proposed algorithm, this condition that jobs use
more processors take longer is not guaranteed.
However, it is strongly encouraged by the 
proportional fairness of the algorithm:
Jobs that would take longer to complete are incentivized
to be able to use more processors.

\underline{Proportional Fairness}:
In the proposed algorithm, jobs' individual computation
times/total required processing are not factored into
the preferences and so have no bearing on the matchings.
Instead, it is the processor utilization ability of 
a job that effects its ranking. This leads to a proportional
fairness in which jobs that are shorter are still given 
a fair amount of processing power so that they will not take
very long. 
On the other hand, jobs that
require more processing power, i.e. would take longer,
are incentivized to be able to use more pocessors than 
jobs that do not take as long.

For a job that would take time to complete 
\(\tau_1 > \tau_2\), where \(\tau_2\) is the
completion time for a second job,
Job 1 would reduce its completion time by an 
absolute amount 
\(\Delta \tau_1 = \tau_1-\frac{\tau_1}{f}\) 
if it could increase its speed by a factor \(f\).
Similarly, for Job 2, 
\(\Delta \tau_2 = \tau_2-\frac{\tau_2}{f}\).
Thus, 
\(\tau_1=\Delta \tau_1(1-\frac{1}{f})\)
and
\(\tau_2=\Delta \tau_2(1-\frac{1}{f})\).
Since 
\(\tau_1 > \tau_2\), we have
\(\frac{\Delta\tau_1}{1-\frac{1}{f}} > \frac{\Delta\tau_2}{1-\frac{1}{f}}\) 
\(\implies \Delta \tau_1 > \Delta \tau_2\).
Therefore, Job 1 has more to gain by increasing its speed by 
a given factor than Job 2 does and so Job 1 has a greater incentive
to be able to use more processors.




\section{Results and Discussion}
Show results and discuss what they represent.

% Inputs Table
\begin{table}[h]
    \resizebox{.485\textwidth}{!}{%
        \begin{tabular}{l|l|l|l|}
            \cline{2-4}
             & \multicolumn{3}{c|}{Comparison} \\ \cline{2-4} 
              & I & II & III \\ \hline
              \multicolumn{1}{|l|}{\begin{tabular}[c]{@{}l@{}}Number of Simulations\end{tabular}} & 500 & 100 & 100 \\ \hline
              \multicolumn{1}{|l|}{\begin{tabular}[c]{@{}l@{}}Number of Computers\end{tabular}} & {[}1,20{]} & {[}1,50{]} & {[}1,20{]} \\ \hline
              \multicolumn{1}{|l|}{\begin{tabular}[c]{@{}l@{}}Number of Jobs\end{tabular}} & {[}1,50{]} & {[}1,100{]} & {[}1,100{]} \\ \hline
              \multicolumn{1}{|l|}{\begin{tabular}[c]{@{}l@{}}Number of Core Types\end{tabular}} & {[}1,5{]} & {[}1,10{]} & {[}1,5{]} \\ \hline
              \multicolumn{1}{|l|}{\begin{tabular}[c]{@{}l@{}}Max. Number of Cores per Type\end{tabular}} & {[}1,25{]} & {[}1,50{]} & {[}1,50{]} \\ \hline
              \multicolumn{1}{|l|}{\begin{tabular}[c]{@{}l@{}}Max. Number of Threads per Job\end{tabular}} & {[}1,50{]} & {[}1,200{]} & {[}1,50{]} \\ \hline
              \multicolumn{1}{|l|}{Speed Ratios} & {[}1,100{]} & {[}1,200{]} & {[}1,200{]} \\ \hline
          \end{tabular}
      }
      \newline
      \caption{Randomized Inputs to Comparisons}
      \label{inputs}
  \end{table}
% End Inputs Table

% Comparison Table I
\begin{table}[h]
\resizebox{.485\textwidth}{!}{%
    \begin{tabular}{l|l|l|l|l|}
        \cline{2-5}
         & S. M. & D. A. 1 & D. A. 2 & Prop. Alg. \\ \hline
        \multicolumn{1}{|l|}{Avg. Time Score} & 780 & 722 & 942 & 989 \\ \hline
        \multicolumn{1}{|l|}{Percent Jobs Assigned} & 24.92\% & 41.67\% & 37.1\% & 34.49\% \\ \hline
        \multicolumn{1}{|l|}{Avg. Avail. Threads Used} & 71.41\% & 78.96\% & 65.57\% & 64.83\% \\ \hline
        \multicolumn{1}{|l|}{Avg. Cores Used} & 63.63\% & 87.94\% & 98.57\% & 98.53\% \\ \hline
        \multicolumn{1}{|l|}{Avg. Resources Score} & 54.49\% & 62.41\% & 50.02\% & 49.5\% \\ \hline
        \multicolumn{1}{|l|}{\begin{tabular}[c]{@{}l@{}}Avg. Time Score Change\\ from Simple Matching\end{tabular}} & 0.00\% & -7.496\% & 20.82\% & 26.78\% \\ \hline
    \end{tabular}
    }
    \newline
    \caption{Comparison I}
    \label{compI}
\end{table}
% End Comparison Table I

% Comparison Table II
\begin{table}[h]
    \resizebox{.485\textwidth}{!}{%
        \begin{tabular}{l|l|l|l|l|}
            \cline{2-5}
             & S. M. & D. A. 1 & D. A. 2 & Prop. Alg. \\ \hline
             \multicolumn{1}{|l|}{Avg. Time Score} & 8668 & 8185 & 10524 & 9161 \\ \hline
             \multicolumn{1}{|l|}{Percent Jobs Assigned} & 57.93\% & 75.15\% & 74.45\% & 74.63\% \\ \hline
             \multicolumn{1}{|l|}{Avg. Avail. Threads Used} & 88.65\% & 85.81\% & 85.00\% & 79.91\% \\ \hline
             \multicolumn{1}{|l|}{Avg. Cores Used} & 72.25\% & 87.51\% & 99.21\% & 86.50\% \\ \hline
             \multicolumn{1}{|l|}{Avg. Resources Score} & 67.09\% & 66.49\% & 61.90\% & 53.64\% \\ \hline
             \multicolumn{1}{|l|}{\begin{tabular}[c]{@{}l@{}}Avg. Time Score Change\\ from Simple Matching\end{tabular}} & 0.00\% & -5.57\% & 21.41\% & 5.69\% \\ \hline
         \end{tabular}
     }
     \newline
     \caption{Comparison II}
     \label{compII}
\end{table}
% End Comparison Table II

% Comparison Table III
\begin{table}[h]
    \resizebox{.485\textwidth}{!}{%
        \begin{tabular}{l|l|l|l|l|}
            \cline{2-5}
             & S. M. & D. A. 1 & D. A. 2 & Prop. Alg. \\ \hline
             \multicolumn{1}{|l|}{Avg. Time Score} & 2742 & 2541 & 3724 & 3516 \\ \hline
             \multicolumn{1}{|l|}{Percent Jobs Assigned} & 48.12\% & 68.78\% & 57.06\% & 59.73\% \\ \hline
             \multicolumn{1}{|l|}{Avg. Avail. Threads Used} & 89.65\% & 95.17\% & 92.89\% & 84.35\% \\ \hline
             \multicolumn{1}{|l|}{Avg. Cores Used} & 47.48\% & 86.52\% & 89.93\% & 88.44\% \\ \hline
             \multicolumn{1}{|l|}{Avg. Resources Score} & 72.73\% & 80.30\% & 74.21\% & 72.60\% \\ \hline
             \multicolumn{1}{|l|}{\begin{tabular}[c]{@{}l@{}}Avg. Time Score Change\\ \\ from Simple Matching\end{tabular}} & 0.00\% & -7.32\% & 35.79\% & 28.21\% \\ \hline
         \end{tabular}
     }
     \newline
     \caption{Comparison III}
     \label{compIII}
\end{table}
% End Comparison Table III

\section{Related Work}
Variations of this work can be found, and they have been extended to handle problems in multiple areas of virtualization and cloud computing.  These problem include topics such as virtual machine co-scheduling, general networking situations were defining utility functions may be difficult, VM migration in cloud computing, distributed loads for VMs, and VM shuffling for congestion reasons.

In “Seen As Stable Marriages,” the authors prepared the background for matching theory to be applied to networking problems, such as ones where utilities difficult or impossible to find.  Instead of trying to pursue optimality, they aimed for stability.  They developed a possible solution for non-centralized coordination between ISP’s in an ISP peering example.  The simplicity that Matching Theory affords, along with privacy benefits, makes this approach interesting.  However, the issue with their work is that it is polarized towards the proposing side of the two parties.  

In their next work, “Egalitarian Stable Matching for VM Migration in Cloud Computing,” the authors sought to reduce the polarization issues they had in the previous paper, similar to what was done in this paper.  The problem they examined was server maintenance scenario, where VM migration is triggered by periodic upgrades, maintenances, and hardware failures.  They introduce a current match dissatisfaction score, a metric for each agent, which was used to encourage a more egalitarian solution.  This dissatisfaction score was derived from how dissatisfied each agent was in its current pair.  The results from this work showed that the overall dissatisfaction was reduced while keeping the stability matching theory offers.  The main issue with this work is that their algorithm performed poorly when the total quotas of the servers was close to the total number of migrating VMs.



\section{Conclusion}
The conclusion goes here.

\section{Future Work}
This paper studies an interesting extension of the college admissions
algorithm where the institutions have multiple quotas of different types
and the applicants can fill any of a particular instution's quotas.
Other, similar modifications to the college admissions algorithm 
may be worthy of interest such as having each applicant only able fill certain 
quota types rather 
simply preferring some types over others.
The proposed algorithm of this paper could also be extended to consider 
scenarios that are complicated by software licensing restrictions
or the need to factor in job length and complexity.
These considerations could lead to more realistic approximations 
so that the algorithm can more practically lend itself to 
real-world implementation.

\section{Individual Contributions}
Kristen Hines formulated the problem, 
and tested the proposed algorithm against other 
approaches which she implemented along with the 
college admissions algorithm to produce the 
results of the project.
She also researched the background literature related to the 
project and contributed to this written report.

Ferdinando Romano formulated the problem,
and developed and implemented the proposed algorithm.
He developed other implementations against which the
proposed algorithm was tested and
he also derived the results about the algorithm's properties 
relating to termination, stability, and optimality
and contributed to this written report.







% use section* for acknowledgement


% trigger a \newpage just before the given reference
% number - used to balance the columns on the last page
% adjust value as needed - may need to be readjusted if
% the document is modified later
%\IEEEtriggeratref{8}
% The "triggered" command can be changed if desired:
%\IEEEtriggercmd{\enlargethispage{-5in}}

% references section

% can use a bibliography generated by BibTeX as a .bbl file
% BibTeX documentation can be easily obtained at:
% http://www.ctan.org/tex-archive/biblio/bibtex/contrib/doc/
% The IEEEtran BibTeX style support page is at:
% http://www.michaelshell.org/tex/ieeetran/bibtex/
%\bibliographystyle{IEEEtran}
% argument is your BibTeX string definitions and bibliography database(s)
%\bibliography{IEEEabrv,../bib/paper}
%
% <OR> manually copy in the resultant .bbl file
% set second argument of \begin to the number of references
% (used to reserve space for the reference number labels box)
\begin{thebibliography}{1}

\bibitem{IEEEhowto:GaleShap}
D.~Gale and L.~S. Shapley, "College Admissions and the Stability
of Marriage," \emph{Amer. Math. Mon.}, vol. 69, no. 1, pp. 9-14, 1962.

\bibitem{IEEEhowto:RothSoto}
A.~E. Roth and M.~Sotomayor, \emph{Two-sided Matching: A Study in
Game Theoretic Modeling and Analysis}, ser. Econometric Society
Monograph.  Cambridge University Press, 1990, no. 18.

\bibitem{IEEEhowto:Roth}
A.~E. Roth, "Deferred Acceptance Algorithms: History, Theory,
Practice, and Open Questions," \emph{Int. J. Game Theory},
vol. 36, pp.537-569, 2008.

\bibitem{IEEEhowto:Roth2}
A.~E. Roth, "Stability and Polarization of Interest in Job
Matching," \emph{Econometrica}, vol. 53, pp.47-57, 1984.

\bibitem{IEEEhowto:Manlove}
D.~F. Manlove, R.~W. Irving, K.~Iwama, S.~Miyazaki, and Y.~Morita,
 "Hard Variants of Stable Marriage," \emph{Elsevier Theoretical Computer Science},
vol. 276, pp.261-279, 2002.

\bibitem{IEEEhowto:Jiang}
"Analysis and Approximation of Optimal Co-Scheduling on Chip Multiprocessors"

\bibitem{IEEEhowto:Xu}
"Seen As Stable Marriages"

\bibitem{IEEEhowto:Xu2}
"Egalitarian Stable Matching for VM Migration in Cloud Computing"

\bibitem{IEEEhowto:Zou}
"VirtualKnotter:  Online virtual machine shuffling for congestion resolving in
virtualized datacenter"


\bibitem{IEEEhowto:Dhillon}
"Virtual Machine Coscheduling: A Game Theoretic Approach"

\bibitem{IEEEhowto:Adolphs}
"Distributed Selfish Load Balancing with Weights and Speeds"


\end{thebibliography}




% that's all folks
\end{document}


